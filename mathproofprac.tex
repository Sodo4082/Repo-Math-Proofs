\documentclass{article}
\usepackage[utf8]{inputenc}
\usepackage{amsmath,amsthm}

\title{Metric Space Q Practice}
\author{Raunak Gupta}
\date{June 2022}


\begin{document}

\maketitle

\section{Questions from Simmons section 9}

\proof Q2\\

Given, $d(x,y) \leq d(x,z) + d(y,z)$ \\
Put x=y , then $0 \leq 2d(y,z)$, Therefore $d(y,z) \geq 0$\\
Since y,z are arbitrary variables, $d(x,y) \geq 0$.\\
Put x=z , then $d(z,y) \leq 0 + d(y,z)$, Therefore $d(z,y) \leq  d(y,z)$\\
Again using the triangle inequality we can write $d(y,z) \leq d(x,y)+ d(x,z)$ and if x=z then we 
have $d(y,z) \leq  d(z,y)$, therefore using the above inequality, d(z,y)= d(y,z)\\
Therefore the given metric d is a metric on X.
\proof Q 3\\
X,d satisfy the three conditions:
$d(x,y) \geq 0$ and $x=y \implies d(x,y)=0,
d(x,y) = d(y,x)$ and $d(x,y) \leq d(x,z) + d(z,y)$\\\\
This is a pseudo metric ,example of a pseudo metric which is not a metric?

Example d(x,y) = 0 $\forall$ x and y\\
Part 2: \emph{doubt}\\
\proof Q 6\\
$I \subseteq \mathbb{R}$
Given: I is an interval
To show: it is non empty and contains all points between any two of its points.

If I is empty then it has no points and hence is not an interval.




\proof  Q 7. \\
X is a metric space with metric d.\\
$x \in X$ and
$A \subseteq X$\\
To prove: if $A \neq \phi ,d(x,A)\geq 0$ and 
$d(x,A)= \infty  \Longleftrightarrow A = \phi$\\\\
(i) To prove If  $d(x,A)=\infty$ then A is empty\\
Since by definition $d(x,y) \geq 0$ therefore $,d(x,A)\geq 0$\\
By definition, $d(x,A) = inf \{d(x,a) : a \in A\}$ 

We know that inf(A) is $\infty$ when A is empty, therefore d(x,A) is empty which only means that the ddomain of the function d(x,A) is empty . \\\\
(ii) To Prove: When A is empty then $d(x,A)=\infty$\\
This is trivially true from the fact that inf(A) is $\infty$ when A is empty, therefore d(x,A) is empty. This is possible only when the domain of d(x,A) is empty. Since x is a given point, A is empty.

\proof Q 8\\
X is a metric space with metric d and a is a subset of X.\\
To prove : if is A is non empty ,d(A) is a non -negative extended real number. 
\\\\
By definition $d(A) \geq 0 $ since it is a metric.\\

(ii) (a) To prove: d(A)= $-\infty$ then A is empty\\
By definition d(A) = sup ( d(x,y) where x,y $\in A$  )
\\
We know that sup A where A is any set is $-\infty$ when A is empty. Therefore d(x,y) is empty. This is possible when there are no points in A, hence A is empty\\\\
(b) To prove:  A is empty then d(A)= $-\infty$ \\
When A is empty the domain of d(x,y) is empty and since sup(A) is $-\infty$ when A is empty therefore d(A) is  $-\infty$.


% ------------------------------------------------------------------------------



\newpage

\section{Questions From Section 10}

\proof Q1\\

To prove: if x and y are distinct points in X then $\exists$ a disjoint pair of open spheres each of which is centered on one of the points.\\

Open sphere: $S_r(x) = \{x: d(x,x') < r\}$

ince two points, let x and y are distinct d(x,y) >0. Therefore we can define a radius\\
$2r_o< d(x,y)$ and make two open spheres :\\

$S_{r_o}(x) = \{x: d(x,x') < r_o\}$ and $S_{r_o}(x) = \{x: d(y,x'') < r_o\}$\\

Therefore making two disjoint open spheres

\proof Q2

To Prove: If $ \{x\} $is a singleton subset of X then show that $ \{x\}' $ is open
(ii) Show that A' is open if A is any finite subset of X

Open set: $\forall x \in A ,\exists S_r(x) = \{x: d(x,x') < r\}\subseteq A$

{x}' = X - {x}


Open set: $\forall x \in X - {x} ,\exists S_r(x) = \{x: d(x,x') < r\}\subseteq X - {x}$

doubt --------------

\proof Q3

To prove: A is a subset of X with diameter less than r which intersects with $ S_r(x)$. Prove that A is a subset of $ S_{2r}(x) $  

This means that, let $x \in A \cap $ S_{2r}(x) $  $






\end{document}

