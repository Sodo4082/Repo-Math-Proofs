\documentclass{article}
\usepackage[utf8]{inputenc}
\usepackage{amsmath,amsthm}

\title{Metric Space Q Practice}
\author{Raunak Gupta}
\date{June 2022}


\begin{document}

\maketitle

\section{Questions from Simmons section 9}

\proof Q2\\

Given, $d(x,y) \leq d(x,z) + d(y,z)$ \\
Put x=y , then $0 \leq 2d(y,z)$, Therefore $d(y,z) \geq 0$\\
Since y,z are arbitrary variables, $d(x,y) \geq 0$.\\
Put x=z , then $d(z,y) \leq 0 + d(y,z)$, Therefore $d(z,y) \leq  d(y,z)$\\
Again using the triangle inequality we can write $d(y,z) \leq d(x,y)+ d(x,z)$ and if x=z then we 
have $d(y,z) \leq  d(z,y)$, therefore using the above inequality, d(z,y)= d(y,z)\\
Therefore the given metric d is a metric on X.
\proof Q 3\\
X,d satisfy the three conditions:
$d(x,y) \geq 0$ and $x=y \implies d(x,y)=0,
d(x,y) = d(y,x)$ and $d(x,y) \leq d(x,z) + d(z,y)$\\\\
This is a pseudo metric ,example of a pseudo metric which is not a metric?

Example d(x,y) = 0 $\forall$ x and y\\
Part 2: \emph{doubt}\\
\proof Q 6\\
$I \subseteq \mathbb{R}$
Given: I is an interval
To show: it is non empty and contains all points between any two of its points.

If I is empty then it has no points and hence is not an interval.




\proof  Q 7. \\
X is a metric space with metric d.\\
$x \in X$ and
$A \subseteq X$\\
To prove: if $A \neq \phi ,d(x,A)\geq 0$ and 
$d(x,A)= \infty  \Longleftrightarrow A = \phi$\\\\
(i) To prove If  $d(x,A)=\infty$ then A is empty\\
Since by definition $d(x,y) \geq 0$ therefore $,d(x,A)\geq 0$\\
By definition, $d(x,A) = inf \{d(x,a) : a \in A\}$ 

We know that inf(A) is $\infty$ when A is empty, therefore d(x,A) is empty which only means that the ddomain of the function d(x,A) is empty . \\\\
(ii) To Prove: When A is empty then $d(x,A)=\infty$\\
This is trivially true from the fact that inf(A) is $\infty$ when A is empty, therefore d(x,A) is empty. This is possible only when the domain of d(x,A) is empty. Since x is a given point, A is empty.

\proof Q 8\\
X is a metric space with metric d and a is a subset of X.\\
To prove : if is A is non empty ,d(A) is a non -negative extended real number. 
\\\\
By definition $d(A) \geq 0 $ since it is a metric.\\

(ii) (a) To prove: d(A)= $-\infty$ then A is empty\\
By definition d(A) = sup ( d(x,y) where x,y $\in A$  )
\\
We know that sup A where A is any set is $-\infty$ when A is empty. Therefore d(x,y) is empty. This is possible when there are no points in A, hence A is empty\\\\
(b) To prove:  A is empty then d(A)= $-\infty$ \\
When A is empty the domain of d(x,y) is empty and since sup(A) is $-\infty$ when A is empty therefore d(A) is  $-\infty$.


% ------------------------------------------------------------------------------



\newpage

\section{Questions From Section 10}

\proof Q1\\

To prove: if x and y are distinct points in X then $\exists$ a disjoint pair of open spheres each of which is centered on one of the points.\\

Open sphere: $S_r(x) = \{x: d(x,x') < r\}$

ince two points, let x and y are distinct d(x,y) >0. Therefore we can define a radius\\
$2r_o< d(x,y)$ and make two open spheres :\\

$S_{r_o}(x) = \{x: d(x,x') < r_o\}$ and $S_{r_o}(x) = \{x: d(y,x'') < r_o\}$\\

Therefore making two disjoint open spheres  $\qedsymbol$

\proof Q2

To Prove: If $ \{x\} $is a singleton subset of X then show that $ \{x\}' $ is open
(ii) Show that A' is open if A is any finite subset of X

Open set: $\forall x \in A ,\exists S_r(x) = \{x: d(x,x') < r\}\subseteq A$

{x}' = X - {x}


Open set: $\forall x \in X - {x} ,\exists S_r(x) = \{x: d(x,x') < r\}\subseteq X - {x}$

\emph{doubt}

\proof Q3

To prove: A is a subset of X with diameter less than r which intersects with $ S_r(x)$. Prove that A is a subset of $ S_{2r}(x) $  

Let an element $a \in A \cap  S_{2r}(x) $ \\
Since the radius of the open circle is r we can say that sup $d(a,x) \leq r$\\ 

Also, if $z \in A$  then $d(a,z) < r$\\
By trangle in equality:\\
$d(x,z) \geq d(x,a) + d(a,z)$\\
$d(x,z) < r + r$ Since $d(a,z) < r$ \\ 
$d(x,z) < 2r$\\

Therefore for an open circle $ S_{2r}(x)$ with center x, $A \subseteq  S_{2r}(x) $ 
$\qedsymbol$



\proof Q.7.\\
To Prove: $Y \subseteq X$ and $A \subseteq Y$ 
Show A is opne as a subset of Y $\doublearrow$ it is the intersection with Y of a set which is open in X.\\
Proof: (i) given A is open as subset of Y then it is the intersection of Y with an open subset of X.\\

For every point x in A there exists an open sphere with radius r such that $ S_r(x) \subseteq A $\\


\proof Q8. \\
1. $Int(A)=x: x\in A$ and  $S_r(x) \subseteq A$ for some r\\ 
Since by a previous theorem we know that an open set is a union of open spheres.
Since the open sets are subsets of A they must be uninons of open spheres of A.\\
But, Int(A) is precisely the set of all points at which we can define an open 
sphere. Therefore all the open subsets of A are subsets of Int(A).

2. (i) A is open then $\forall x \in A , \exists r: S_r(x) \subseteq A $\\
And since Int(A) is the set of all points of A at which an open sphere can be 
defined, Int(A)=A since for A at every point an open sphere can be defined.
(ii) If A=Int(A) then since Int(A) is open, A is open

3. From (i) 
$\qedsymbol$

\proof Q.10. \\
A and B $\subseteq$ X\\
(a) $Int(A) \cup Int(B) \subseteq Int (A \cup B)$\\
(b) $Int(A) \cap Int(B) = Int(A\cap B)$


(a)Proof: Int(A) and Int(B) are open sets.\\
Union of open sets are open hence $Int(A) \cup Int(B)$ is open.Since it is open
we can define an open sphere at every point of this set. This will be nothing 
but all the points of the set $A \cup B$ at which an open sphere can be defined.\\
That is if $z \in Int(A) \cup Int(B)$ then $z \in Int(A \cup B)$\\
Therefore: $Int(A) \cup Int(B) \subseteq Int( A \cup B)$\\
(b)Proof: Since the intersection of finite open sets is open, $Int(A) \cap Int(B)$ is also open. 
Int(A) is set of all the points at which an open sphere can be defined. Therefore $Int(A) \cap Int(B)$ is the set of all the common points of A and B at which an open sphere can be defined hence:\\
 $z \in Int(A) \cap Int(B)$ then $z \in Int(A \cap B)$\\
Therefore: $Int(A) \cap Int(B) \subseteq Int( A \cap B)$\\
(ii) Similarly if $z \in  Int( A \cap B) $ then $z \in Int(A) \cap Int(B)$\\
Therefore, $Int( A \cap B) \subseteq Int(A) \cap Int(B) $\\
Hence: $Int( A \cap B) = Int(A) \cap Int(B) $\



\end{document}

